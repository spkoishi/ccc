\documentclass{ctexart}

\usepackage{graphicx}
\usepackage{amsmath}

\title{作业一:给出黎曼可积和勒贝格可积的定义, 并分析二者的区别}


\author{蔡聪聪 \\ 信息与计算科学 3180102279}

\begin{document}

\maketitle


为了对数学分析中某些结果加深理解,这里就一维情形考虑有限区间[a,b]上f(x)的积分,对f上的黎曼积分(简称R积分)与勒贝格积分(简称L积分)进行若干比较
\section{R可积和L可积的定义}
\subsection{R可积的定义}
设f是定义在[a,b]上的有界函数,区间[a,b]的任一分划\\
\begin{equation*}
  x_0=a<x_1<x_2<\cdots<x_n=b
\end{equation*}\\
将[a,b]分成n个部分,在每个小区间$[x_{i},x_{i+1}]$上任取一点 $\xi_i,i=0,1,\cdots,n-1$,作和\\
\begin{equation}
  \sigma=\sum\limits_{i=0}^{n-1}f(\xi_i)(x_{i+1}-x_i).
\end{equation}\\
令$\lambda=max(x_{i+1}-x_i)$.如果对区间任意的划分与$\xi_i$的任意取法,当$\lambda\xrightarrow[]{}0$时,$\sigma$趋于有极限的I,则称它为f在[a,b]上的R积分,记为\\
\begin{equation}
I=(R)\int_a^bf(x)dx.
\end{equation}\\

\subsection{L可积的定义}
设f是定义在可测集D上的可测函数,对每一$x\in D$,令\\
\begin{equation}
f_{+}(x)=max\left\{0,f(x)\right\},f_{-}(x)=max\left\{0,-f(x)\right\}
\end{equation}\\
则$f_{+}$和$f_{-}$分别称为函数f的正部和负部,它们都是非负可测函数并且\\
\begin{equation}
f(x)=f_{+}-f_{-},|f(x)|=f_{+}+f_{-}.
\end{equation}\\
今若$\int_Df_{+}dx$和$\int_Df_{-}dx$不同时为$\infty$,则f在D上的勒贝格积分定义为\\
\begin{equation}
\int_Dfdx=\int_Df_{+}dx-\int_Df_{-}dx.
\end{equation}\\
此外当$\int_Dfdx$有限时,称f在D上L可积,并记为$f\in L(D)$.

\section{R积分与L积分比较}
从某些极限过程来看,L积分较R积分优越些.\\
举傅里叶级数的逐项积分问提来做进一步说明.在数学分析中,这个问题是不易讲的透彻的.现在用L积分观点来讨论.假定f(x)是以2$\pi$为周期的L可积函数,那么它有傅里叶展式\\
\begin{equation}\label{ep::a2}
f(x)\sim\dfrac{a_0}{2}+\sum\limits_{n=1}^{\infty}(a_ncosnx+b_nsinnx)
\end{equation}\\
其中\\
\begin{gather*}
  a_n=\dfrac{1}{\pi}\int_{-\pi}^{\pi}f(x)cosnxdx, n=0,1,2,\cdots,\\
  b_n=\dfrac{1}{\pi}\int_{-\pi}^{\pi}f(x)sinnxdx, n=1,2,\cdots,
\end{gather*}\\
所写展开式(\ref{ep::a2})并不表示级数收敛,但是,可积函数f(x)的傅里叶展开式却可以逐项积分.就是说,有等式\\
\begin{equation}\label{ep::a3}
\int_{\alpha}^{\beta}f(t)dt=\int_{\alpha}^{\beta}\dfrac{a_0}{2}dt+\sum\limits_{n=1}^{\infty}\int_{\alpha}^{\beta}(a_ncosnt+b_nsinnt)dt
\end{equation}\\
成立,其中$[\alpha,\beta]$是$[-\pi,\pi]$的任意子区间,所有积分均指勒贝格积分.\\
其实,引进区间特征函数$\varphi=\chi_{[\alpha,\beta]}(x)$来讨论,它限制在$[-\pi,\pi]$上定义,我们将它延拓到$(-\infty,\infty)$使其有周期$2\pi$,并约定保持函数记号不变.那么$\varphi(x)$有傅里叶展开式\\
\begin{equation}\label{ep::a4}
\varphi(x)\sim\dfrac{A_0}{2}+\sum\limits_{n=1}^{\infty}(A_ncosnx+B_nsinnx)
\end{equation}\\

令级数(\ref{ep::a4})的部分和为$\varphi_n(x)$,则几乎处处有$\varphi(x)=\lim\limits_{n \rightarrow \infty}\varphi_n(x)$\\
不难验明,要证的等式(\ref{ep::a3})化为等式\\
\begin{equation}\label{ep::a5}
\int_{-\pi}^{\pi}f(x)\varphi(x)dm=\lim\limits_{n \rightarrow\infty}\int_{-\pi}^{\pi}f(x)\varphi_{n}(x)dm
\end{equation}\\
现知函数列$\left\{f(x)\varphi_{n}(x)\right\}_{n \in N}$的极限几乎处处为$f(x)\varphi(x)$,而且可以用初等方法证明$\varphi_{n}(x)$是一致有界的,$\varphi_{n}(x)\le C(n \in N)$.\\

这样,函数序列$\left\{f(x)\varphi_{n}(x)\right\}_{n \in N}$有可积的控制函数$C|f(x)|$.根据勒贝格控制收敛定理,有(\ref{ep::a5})成立.由此可见,用勒贝格积分解决傅里叶级数逐项积分问题是相当有效的.

\end{document}
